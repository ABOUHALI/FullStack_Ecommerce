\makeatletter
\@ifclassloaded{scrartcl}{
  \maketitle{}
}{
  % die selbst entworfene Titelseite, die evtl.  automatisch angepasst
  % werden kann
  \thispagestyle{empty}
  \begin{titlepage}

      % die Leerzeilen bedeuten immer das Ende bzw. den Beginn eines neuen
      % Absatzes, ohne eine Leerzeile wird der laufende Text fortgesetzt

      % die Ortsangabe soll einheitlich rechtsbündig gesetzt werden, dafür
      % wird die Angabe in einen Kasten (\parbox[position][height][inner
      % position]{width}{text}) gepackt, der passend verschoben wird

      % die Box soll rechtsbündig erscheinen, um waagerechten Leerraum zu
      % erhalten, benutzt man \hspace{length}. length kann eine variable
      % Größe ein, mit der Platz aufgefüllt wird: \fill

      % eine geratene Größe für eine Breite ist manchmal ausreichend,
      % meistens aber nicht
      \settowidth{\myWidth}{\sffamily\large Angewandte Mathematik und
        Informatik}
      % ^ die Auswahl und die Größe des Fonts sind wichtig, sonst stimmen
      % Größen nicht überein
      \hspace{\fill}\parbox{\myWidth}{
        % da mehrere Zeilen in einem anderen Font gesetzt werden sollen,
        % benutzt man Befehle, die gruppenintern gelten
        % \scshape
        %   Kapitälchen (sc: small caps), alle Buchstaben werden groß
        %   geschrieben, die Großbuchstaben ein bisschen größer, entspricht
        %   nicht dem Corporate Design der FH
        \sffamily % serifenloser Font
        \raggedleft % rechtsbündiger Satz
        \large Fachhochschule Aachen Campus Jülich

        Medizintechnik und Technomathematik

        Angewandte Mathematik und Informatik }
      \begin{tikzpicture}[remember picture, overlay]
          \node [xshift=15mm, yshift=-51mm](a) at (current page.north west)
          {%
            \includegraphics{fh_logo_links}};
      \end{tikzpicture}

      % eine gute Einteilung von Titelseiten erfolgt nach dem
      % Drittelprinzip oder nach dem Golgenen Schnitt Drittelprinzip (um
      % 90° gedreht): 2/3        1/3
      %               +----------+-----+
      %               |          |     |
      %               +----------+-----+
      %
      % Goldener Schnitt (um 90° gedreht): a          b
      %                                    +----------+-----+
      %                                    |          |     |
      %                                    +----------+-----+
      % mit a:b = (a+b):a

      \vspace{160pt}%
      \rule{\textwidth}{2pt}

      \begin{center}
          \Huge \textsf{\bfseries Webshopping mit Spring Boot und React}

          % um senkrechten Leerraum zu erhalten, beginnt man den Absatz mit
          % einem \vspace{length}. length ist eine Größe, die in einer
          % beliebigen Einheit angegeben werden kann, normalerweise in pt
          % pt: point, 1 pt ist der 1/72,27 Teil eines inches (1 in = 2,54
          % cm), 10 pt entsprechen ungefähr 3 mm.
          \vspace{20pt}%
          \Large  eine PraxisProjekt von Ayman~ABOUHALI
          % ~: bedeutet einen nicht umbrechbaren Zwischenraumes wäre ein
          % schwerer typographischer Fehler, ...  N.~N.: aus dem
          % Lateinischen - nomen nominandum: ein noch zu benennender Name -
          % nomen nescio: ich weiß den Namen (noch) nicht
      \end{center}

      \rule{\textwidth}{2pt}

      \vspace{15pt}
      \begin{center}
          \today
      \end{center}

  \end{titlepage}
}
\makeatother
