% Trick: verhindere zweifaches Einlesen dieser Datei
\csname SwitchCfgLoaded\endcsname   % Dies ist die Konstruktion des
                                    % Befehls \SwitchCfgLoaded
\let\SwitchCfgLoaded\endinput

\typeout{This is switch-cfg.tex, the single source publishing configuration
file}

% boc: define flags
% flags, die das Formatierungsziel beschreiben
\newboolean{isBook}             % alle flags sind mit
\newboolean{isArticle}          % false initialisiert
\newboolean{isArticleTC}
\newboolean{isReport}
\newboolean{isPresentation}
\newboolean{isPoster}
\newboolean{useBeamerarticle}
% eoc: define flags

% boc: set flags
% mache die folgenden Tests formatierbar
\makeatletter                   % damit ist "@" ein Buchstabe,
                                % ein Befehl darf nun ein "@"
                                % enthalten!
\@ifclassloaded{scrbook}{       % formatiere als Buch
  \setboolean{isBook}{true}
}{                              % formatiere nicht als Buch
  \@ifclassloaded{scrreprt}{    % formatiere als Report
    \setboolean{isReport}{true}
  }{                            % formatiere nicht als Report
    \@ifclassloaded{scrartcl}{  % formatiere als Artikel
      \setboolean{isArticle}{true}
      \@ifclasswith{scrartcl}{twocolumn}{ % zusätzlich:
                                % zweispaltiger Artikel
        \setboolean{isArticleTC}{true}
      }{}                       % einspaltiger Artikel
    }{                          % formatiere nicht als Artikel
      \@ifclassloaded{beamer}{  % formatiere als Präsentation
        \setboolean{isPresentation}{true}
        \@ifpackageloaded{beamerposter}{ % zusätzlich:
                                % formatiere als Poster
          \setboolean{isPoster}{true}
          \defbibheading{bibliography}[]{}
        }{}
      }{}                       % formatiere nicht als
                                % Präsentation
    }
  }
}
\@ifpackageloaded{beamerarticle}{ % zusätzlich:
  % benutze Paket zum Wegdefinieren vom beamer-Befehlen
  \setboolean{useBeamerarticle}{true}
}{}
\makeatother    % @ darf nicht mehr in einem Befehl auftauchen
% eoc: set flags

% hier stehen Einstellungen, die für fast alle Dokumentklassen gebraucht
% werden, Ausnahmen werden weiter unten definiert
\def\sFactor{1.0}               % Variablen werden mittels des
                                % Original-TeX-Kommandos
                                % \def\<cmd-name>{} gesetzt.
                                % \<cmd-name> gilt gruppenweit
                                % und kann durch ein weiteres
                                % \def einfach umgesetzt werden

% boc: use flags
% kombinierte Fälle werden hier abgehandelt, dadurch wird eine
% tiefe Schachtelung vermieden
\ifthenelse{\boolean{isArticleTC} \or \boolean{isPoster}}{
  \usepackage{multicol}             % Um den Text nicht zu breit
                                    % werden zu lassen, wird er
                                    % in einer Umgebung
                                    % "multicols" geschrieben,
                                    % dies erzeugt zwei oder
                                    % mehr Spalten
  \setlength{\columnsep}{1em}       % Platz zwischen den Spalten
  \setlength\columnseprule{0.1em}   % Es wird eine optische
                                    % Trennung durch eine
                                    % senkrechte Linie erzeugt.
}{}
\ifthenelse{\not \boolean{isPresentation}}{
  \ifthenelse{\boolean{useBeamerarticle}}{ % für die Vorlesung ist dieser
    \renewcommand{\frametitle}[1]{} % Befehl unbekannt, in den Projekten
  }{                             % nicht (wg. beamerarticle)
    \newcommand{\frametitle}[1]{}   % Deshalb muss hier mit
  }                              % \newcommand{cmd}[args]{def}
                                 % gearbeitet werden, in den
                                 % Projekten mit
                                 % \renewcommand{cmd}[args]{def}
                                 % Allgemein kann das gelöst werden,
                                 % indem getestet wird, ob ein Paket,
                                 % das nur für die Vorlesung
                                 % gebraucht wird, geladen ist
}{}
% eoc: use flags

% boc: distinguish classes
\ifthenelse{\boolean{isBook}}{
  % dies ist der Standardfall, hier braucht nichts weiter
  % gemacht zu werden
}{
  % kein Buch, deshalb: definiere Kommandos weg, die nicht
  % in einem Buch auftreten können
  %
  % \let \Kommando \neuesKommando macht Folgendes:
  % wenn der TeX-Compiler auf \Kommando trifft, wird statt dessen
  % \neuesKommando ausgeführt
  \let\frontmatter\relax    % \relax ist das "leere" Kommando, das
                            % immer ohne Auswirkungen aufgerufen
                            % werden kann
  \let\mainmatter\relax
  \let\backmatter\relax
  \ifthenelse{\boolean{isReport}}{
  }{
    % kein Buch, kein Report => kein \chapter, also herunterstufen
    % der Gliederungshierarchie, kann fast beliebig fortgesetzt
    % werden...,
    \let\chapter\section
    \let\section\subsection
    \let\subsection\subsubsection

    \ifthenelse{\boolean{isArticle}}{
      % ------- Beginn Spezialfall: 2-spaltiger Artikel ----------
      \ifthenelse{\boolean{isArticleTC}}{
        % hier müssten Faktoren auftreten, die Bilder auf
        % \columnwidth skalieren
      }{}
      % ------- Ende Spezialfall 2-spaltiger Artikel -------------
    }{
      \ifthenelse{\boolean{isPresentation}}{
        \AtBeginSubsection[]{%   Am Anfang einer \subsection wird
                             %   eine eigene Folie generiert
          \begin{frame}<beamer|handout>%
              \begin{block}{}
                  \insertsubsectionhead
              \end{block}
          \end{frame}
        }

        % ------- Beginn Spezialfall Poster ---------------------
        \ifthenelse{\boolean{isPoster}}{

          \def\sFactor{2.83}    % = \sqrt{2}^3,
                                % wg. DIN A4 -> DIN A0/2

          \renewcommand{\maketitle}{% Umdefinieren des bekannten
                                % Kommandos:
                                % \maketitle darf keine eigene Seite
                                % erzeugen!
            \begin{center}%
                \LARGE\inserttitle\\% Zugriff auf die internen Daten
                \normalsize\insertinstitute%
            \end{center}%
          }
          \def\onslide<+->{}    % korrigiere \onslide-Fehler für
                                % Poster
        }{}
        % ------- Ende Spezialfall Poster -----------------------

      }{}
    }
  }
}
% eoc: distinguish classes

\defbeamertemplate*{footline}{ACUAS theme}{%
  \begin{beamercolorbox}[wd=\paperwidth, ht=0.025\paperheight,
      dp=0.005\paperheight]{palette primary}%
      \begin{pgfpicture}
          \pgfusepath{use as bounding box}
          \pgfsetlinewidth{0.8pt}
          \pgftransformshift{\pgfpointanchor{current page}{south west}}
          \pgftransformshift{\pgfpoint{0.03\paperwidth}{0.05\paperheight}}
          \pgfpathmoveto{\pgfpointorigin}
          \pgfpathlineto{\pgfpoint{0.88\paperwidth}{0pt}}
          \pgfusepath{stroke}
          \pgfrememberpicturepositiononpagetrue
      \end{pgfpicture}%
      \usebeamerfont{author}%
      \hspace{0.03\paperwidth}%
      \hbox to 0pt{%
        \copyright{} \jhfb@published{}~%
        \insertshortauthor
        -- \insertshorttitle{} -- \today{}%
      }%
      \hspace{\fill}%
      \insertframenumber
      \hbox to 0.09\paperwidth{}%
  \end{beamercolorbox}%
  \vskip5pt%
}
